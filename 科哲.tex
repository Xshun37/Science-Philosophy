\documentclass[12pt, a4paper, oneside]{ctexart}
\usepackage{amsmath, amsthm, amssymb, graphicx}
\usepackage[bookmarks=true, colorlinks, citecolor=blue, linkcolor=black]{hyperref}
\usepackage{xcolor}
\usepackage{tocloft}
\usepackage{fancyhdr}
\usepackage{titlesec}
\usepackage{geometry}
\usepackage{fontspec}
\usepackage{enumitem}
\usepackage{tcolorbox}     % 彩色盒子
\usepackage{boondox-cal}
\usepackage{perpage}  % 加载perpage包

\MakePerPage{footnote}  % 脚注每页重新编号
\makeatletter
\@removefromreset{footnote}{section}   % 解除与section的关联
\@removefromreset{footnote}{subsection} % 解除与subsection的关联
\makeatother
\tcbuselibrary{skins}      % 启用皮肤库,支持标题栏设计

\setCJKmainfont[BoldFont=SourceHanSerifSC-Bold.otf]{SourceHanSerifSC-light.otf}
\usepackage{newtxtext}
\renewcommand{\b}{\textbf}
\newcommand{\f}{\footnote}
\newcommand{\s}{\setlength{\parindent}{2em}}
\newcommand{\q}[1]{\begin{questionbox}{}#1\end{questionbox}}
\newcommand{\ans}[1]{\begin{ansbox}{}#1\end{ansbox}}
\newcommand{\quo}[1]{\begin{quote}\small{}#1\end{quote}}
% ===== 颜色定义 =====
\definecolor{titlepink}{RGB}{135, 14, 79}  % 标题粉色
\definecolor{questionbg}{RGB}{245, 240, 255}  % 浅紫色背景
\definecolor{titleblue}{RGB}{41, 98, 255}     % 小标题颜色
\definecolor{ansb}{RGB}{214, 248, 225}
\definecolor{ansback}{RGB}{30, 120, 36}

% ===== 提问模块定义 =====
\newcounter{question}[section]
\renewcommand{\thequestion}{\arabic{question}}
\newtcolorbox[use counter=question]{questionbox}[1][]{
    colback=questionbg,      % 使用预定义的背景色
    colframe=titlepink,      % 使用预定义的边框色
    coltitle=white,          % 标题文字颜色
    title={Q.\thesection.\thequestion},              % 标题内容
    fonttitle=\bfseries, % 标题字体
    enhanced,                % 启用增强功能
    attach boxed title to top left={xshift=5pt,yshift=-10pt}, % 标题栏位置
    boxed title style={      % 标题栏样式
        colback=titlepink,   % 使用预定义的标题栏颜色
        colframe=titlepink,  % 标题栏边框颜色
    },
    rounded corners,
    after skip=1.2em,        % 调整内容区后的间距(兼容所有版本)
    fontupper=\normalsize,   % 内容字体大小
    left=15pt,               % 内容左侧间距
    right=15pt,              % 内容右侧间距
    top=13pt,                % 内容顶部间距
    bottom=12pt,             % 内容底部间距
}

% ===== 回答模块定义 =====
\newcounter{ans}[section]
\renewcommand{\theans}{\arabic{ans}}
\newtcolorbox[use counter=ans]{ansbox}[1][]{
    colback=ansb,      % 使用预定义的背景色
    colframe=ansback,      % 使用预定义的边框色
    coltitle=white,          % 标题文字颜色
    title={A.\thesection.\theans},              % 标题内容
    fonttitle=\bfseries, % 标题字体
    enhanced,                % 启用增强功能
    attach boxed title to top left={xshift=5pt,yshift=-10pt}, % 标题栏位置
    boxed title style={      % 标题栏样式
        colback=ansback,   % 使用预定义的标题栏颜色
        colframe=ansback,  % 标题栏边框颜色
    },
    rounded corners,
    after skip=1.2em,        % 调整内容区后的间距(兼容所有版本)
    fontupper=\normalsize,   % 内容字体大小
    left=15pt,               % 内容左侧间距
    right=15pt,              % 内容右侧间距
    top=13pt,                % 内容顶部间距
    bottom=12pt,             % 内容底部间距
}

% ===== 目录格式设置 =====
% 主标题格式(粉色居中)
\renewcommand{\contentsname}{目录}
\renewcommand{\cfttoctitlefont}{\hfill\Large\bfseries\textcolor{titlepink}}
\renewcommand{\cftaftertoctitle}{\hfill}  % 保持居中

% 前导符和页码样式
\renewcommand{\cftsecleader}{\cftdotfill{\cftdotsep}}  % 点状前导符
\renewcommand{\cftsecpagefont}{\color{red}}       % 章节页码颜色
\renewcommand{\cftsubsecpagefont}{\color{red}}    % 子章节页码颜色
\renewcommand{\cftsubsubsecpagefont}{\color{red}}    % 子章节页码颜色
\renewcommand{\cftdotsep}{1}                           % 点间距
\renewcommand{\cftbeforesecskip}{3pt}                  % 章节垂直间距

% ===== 标题格式设置 =====
% Section标题(居中/粉色序号)
\titleformat{\section}[block]
{\centering\bfseries\huge}
{\color{titlepink}\thesection}  % 粉色序号
{0.5em}                         % 序号标题间距
{\textcolor{titlepink}}        

% 定义subsection编号格式为 "section.阿拉伯数字"
\renewcommand{\thesubsection}{\thesection.\arabic{subsection}}
% 定义subsubsection编号格式为 "subsection.阿拉伯数字"
\renewcommand{\thesubsubsection}{\thesubsection.\arabic{subsubsection}}

% 重置subsection和subsubsection计数器(从1开始)
\setcounter{subsection}{1}
\setcounter{subsubsection}{1}

\titleformat{\subsection}[block]
{\bfseries\Large}
{\color{titlepink}\thesubsection}  % 粉色序号
{0.5em}                         % 序号标题间距
{\textcolor{titlepink}}        

\titleformat{\subsubsection}[block]
{\raggedright\bfseries\large}
{\color{titlepink}\thesubsubsection}  % 粉色序号
{0.5em}                         % 序号标题间距
{\textcolor{black}}

% ===== 超链接设置 =====
\hypersetup{
    linkcolor=red,         % 目录链接墨绿色
    citecolor=blue,        % 引用链接蓝色
    linkcolor=black        % 普通链接黑色
}

% ===== 页眉页脚设置 =====
\pagestyle{fancy}
\fancyhf{}
\chead{\thepage}  % 页眉居中显示页码
\cfoot{\thepage}   % 页脚居中显示页码

% ===== 自定义命令 =====
\newcommand{\dsigma}{d\sigma}  % 积分微分符号统一格式
\newcommand{\dxdy}{dxdy}       % 二重积分微分符号

% ===== 页面布局调整 =====
\geometry{
    top=2.5cm,
    bottom=2.5cm,
    left=2.5cm,
    right=2.5cm,
    headheight=14pt
}

\title{\textbf{科学哲学前沿 - Week 1}}
\author{$\mathbcal{Xshun}$}
\date{2025年7月7日}

\begin{document}

\maketitle
{
\hypersetup{linkcolor=red}
\tableofcontents
}
\section {Monday}
\subsection{《认识论危机、戏剧性叙事与科学哲学》}
\textbf{Epistemological Crises, Dramatic Narrative and the Philosophy of Science, MacIntyre}
\subsubsection{什么是认识论危机?}

一个\b{普通行动者}(Ordinary agents)自我认知的结果往往与实际发生的情形不同,从表面现象(seems)到实质(is)之间的差别让人们开始思考他心问题\f{他心问题:心灵作为私密实体,是否以及如何被他人直接认知。}和归纳的辩护问题\f{归纳的辩护:探讨归纳法在认知层面是否具有合理性和正当性,比如从概率或频率的角度,可以主张归纳的有效性取决于命题间的概率关系。证明归纳正当性的问题被称为“休谟问题”。}。于是人们发现,同样根据个体行为推断其行事风格、思想情感,却可以通过对立的解释\f{笔者的理解:当情况不同时,我们对同一个现象的解释也会不同。比如,当我们与一个人的关系生疏,友善的行为也会被视作是有预谋的讨好。}得到截然不同的结论。然而,由于社会生活建立在我们能理解彼此的行为的基础上,这会使得正常的生活难以为继。

那么,这样的基础\f{即理解彼此的行为。}能以任何形式被辩护吗?

现在,考虑我们是如何共享一种文化的。我们共享一种文化,实质上是共享一种\b{Schemata},我们通过这个框架来规约自己的行为,同时解释他人的行为。可见,理解他人的行为和实施可被理解的行为本质相同,正是通过一个相同的\b{Schemata},我们得以理解和推论他人的行为。

然而,正如前文所述,对同一个现象有不同的系统性解释,也就是说,存在着相互竞争的\b{Schemata},其对同一现象有相互矛盾的描述。这也就是普通行动者所遭遇的\b{认识论危机}(Epistemological Crisis)\f{认识论危机指个体或群体对自身信念系统、知识框架或世界观的根本性怀疑和崩溃。它表现为无法用现有认知工具(如理论、叙事、信仰)来解释经验或现实,导致意义丧失、方向混乱和行动瘫痪。}。

作者在这里引用了\textit{Hamlet}来解释他的观点。从维滕贝格归来的哈姆雷特有多种用于解释埃尔西诺事件的\b{Schemata},而他同时面临着“应当选用哪个Schemata”和“应该相信谁”的问题,前者让他不知道该把什么当做证据,后者让他不知道选用哪个图示,陷入了循环的认识论危机。同样的,这样的认识论危机一定也存在于想要复现哈姆雷特故事的导演和想要理解哈姆雷特故事的文学家心中。

\subsubsection{戏剧性叙事、传统与怀疑论}

首先我们需要定义什么是叙事。
    
当我们思考“究竟发生了什么”时,本质上是在思考\b{“该如何构建叙事?”}。事实上,回忆前文的认识论危机,为了解决它,我们必须构建一个新的叙事,而为了重构叙事,我们需要不断探询现状来反转我们对过去时间的理解,并同时追求叙事的\b{真理性}(Truth)和\b{可理解性}(Intelligibility)。

不幸的,追求真理往往可能破坏一个至今可理解的叙述,因此对两者的追求并非总是协调一致的。当我们逐渐探询真相,就会让自己越不可理解,如果做的极端,就会陷入疯狂。疯狂或死亡会阻止我们对认识论危机的解决,因为认识论危机总是人际关系的危机。如果人们不能对叙事进行重构,共享的叙事缺失,社会范围的认识论危机就会滑向非理性的结局\f{比如,“后真相时代”下,客观事实的重要性被个人理解取代,这正源于社会范围内人际信任的崩塌,共享叙事的缺失,最终导致群体的极化和暴力,而非理性对话。}。而在构建新叙事时,我们发现新的叙事也总是需要被质疑的,甚至衡量真理性,可理解性,理性的标准本身也是需要被质疑的。我们永远无法声称我们掌握了真理,或者说我们是完全理性的。

容易发现,认识论的进步就在于构建更充分的叙事,而认识论危机正是这种重构的契机。随之而来的问题是,认识论的进步是何时开始的?我们是从哪一个叙事出发?

不妨从一个人的幼年开始。在俄狄浦斯(Oedipus)期\f{俄狄浦斯期:弗洛伊德的重要理论,其提出的“俄狄浦斯情结”又称“恋母情结”。一般指儿童的性别认同发展阶段(3-6岁),在这个阶段儿童开始感受到性别认同,对异性父母有更深刻的感受,并且体验到竞争和冲突等情绪。},我们开始受到外部世界的影响并尝试理解它,同时从好的童话故事中学会如何投入社会现实并感知其中的秩序。一个被剥夺了正确类型童话故事的儿童往往在将来选择逃避他难以解释和应对的现实。\f{俄狄浦斯是拉伊俄斯(Laius)和约卡斯塔(Jocasta)的儿子,Laius投奔珀罗普斯(Pelops),诱杀了他的儿子克律西波斯。Pelops诅咒Laius的儿子,也即将来的Oedipus将犯下弑父娶母之罪,于是Laius抛弃了Oedipus。多年后,为了惩治Laius对克律西波斯的罪行,斯芬克斯来到忒拜城,为了击退斯芬克斯,Laius前往特尔斐神庙,却与Oedipus狭路相逢,二人随之发生争斗,Oedipus便在不知情的情况下杀死了自己的父亲,并在进入忒拜城后,回答出了斯芬克斯的谜语,被推崇为国王娶了自己的母亲Jocasta。}

笛卡尔弃绝历史作为通往真理的手段,并认为,他应当从一个什么也不知道的假设出发,直到找到一个能够建立一切的基础性的\b{无预设的第一原则}(Presuppositionless First Principle)。这是一种\b{无背景的怀疑}(Contestless Doubt)。

\q{作者说,脱离原先的叙事以便发现真正的叙事,是认识论的转变点。在这个点上,问题被强加于叙述者,使其无法继续将其用作解释的工具。这里的问题是指什么?}
\ans{目前我的想法是,这个问题就是我们所面临的认识论危机。过去已有的叙事已经崩溃,因此我们不能用它来解决所面临的新的认识论问题。}

笛卡尔抛弃了人类的历史,但是却向我们讲述他认知的历史来作为进行真理探索的媒介。很显然这是冲突的,这一刻发生了认识论危机。作者说: “...All those questions which the child has asked of the teller of fairy tales arise in a new adult form. Philosophy is now set the same task that had once been set for myth(儿童向童话讲述者提出的所有问题都以新的成人形式出现。哲学现在被赋予了曾经神话的任务). ”

而值得注意的是,一个真正相信自己一无所知的人根本不知道应该如何开始真正的怀疑,而笛卡尔式的认识论者却从来不会怀疑自己那一整套的广泛的认识论信念。并且,笛卡尔从来没有质疑自己所使用的法语和拉丁语在表达相同思想上的能力,换言之,他没有质疑他使用的语言中所继承下来的东西,而语言中继承着一套用于整理思想和世界的方式。正如作者所说“...how much of what he took to be the spontaneous reflections of his own mind was in fact a repetition of sentences and phrases from his school textbooks...(他自认为是心灵反思的很多东西,实际上只是对课本上句子和短语的重复)”。

笛卡尔认为,我思故我在,也就是说\b{“正在思考的自我”}的存在是不可被怀疑的。想想看,如果这样,笛卡尔需要考虑如何从“纯粹的理性自我”抵达“外部世界的确定性”。有趣的是,这正与柏拉图,奥古斯丁等人所思考的传统类似:心灵之眼在理性之光下关照其对象。在这一传统下,柏拉图将认知分为知识和感官经验。问题在于“感官经验”作为不完美的事物,是如何能通向作为永恒真理的“知识”的?从一个不可靠的起点到达一个可靠的终点,这正是他们都面临的\b{“Metaphorical Incoherence(隐喻性的不连贯)”}。

如果感官本身被判定为不可靠,我们又该如何得到可靠的确定性知识?这正给了怀疑论者靶子,如果知识的来源(感官)不可信,而知识的形式又无法合理的扎根于来源,那么所谓的“知识”是否只是一种心灵的虚构?笛卡尔的哲学起点就带有怀疑论的印记,并试图通过“我思故我在”证明知识的绝对不可怀疑。问题在于,同样的考虑笛卡尔面临的隐喻性的不连贯,我们会发现:
\begin{enumerate}
    \item \b{笛卡尔循环}。笛卡尔认为“任何我感知得清楚明白的事物都是真的”,因为上帝存在,而不是一个欺骗者。然而为了证明上帝存在,他却必须明确“清楚明白的感知可靠”这个前提。因此笛卡尔的证明涉及到了一个循环论证的谬误;
    \item 笛卡尔循环证明了上帝的担保无效,因此从理性思考到外部世界的桥梁是不可靠的。那么我们就可以质疑外部世界是一场幻觉;
    \item 笛卡尔循环似乎表明理性需要外部担保而又无法获得,也就说,理性的运用总依赖一个无法证明的预设,那么理性的“确定性”是否只是一种自欺欺人?
\end{enumerate}

隐喻性不连贯揭露了西方哲学的深层困境:绝对确定的知识必须预设知识与感官经验的严格分离,但是知识不可能完全脱离感官经验而存在。

作者认为,笛卡尔根本不会意识到在用创造上帝这个行为自己在回应怀疑论者的问题\f{当然,由于笛卡尔循环,这种论证本身也显然不可靠。},因为他自以为自己已经抛弃了一切传统。

在怀疑论的压力下,这种传统关联“seems”和“is”的方式土崩瓦解,面临严重的认识论危机。在那个时代,莎士比亚借助\textit{Hamlet}让我们反思自我的危机,但笛卡尔却切断了认识自我的可能性,发明了一种非历史的,自我认可(self-endorsed)的自我意识用于描述认识论危机。

接下来我们考虑伽利略。在伽利略时代,托勒密的地心说和哥白尼的日心说在天文学体系上出现了冲突。柏拉图认为,天文学研究应当关注理想天体的运动与数学。而托勒密体系天文学的几何单纯性极差,还破坏了亚里士多德对于天体匀速圆周运动的完美性的物理学假说。同时,柏拉图注意和亚里士多德的物理学原则又与牛顿等人关于运动的研究成果不一致。因此,\b{工具主义}的科学哲学流行起来,其认为,理论不必是真是描述世界的,只要他能够准确预测观测现象就可以当做有用的“工具”来使用。

想一想,工具主义带来了什么:人们不再相信存在描述世界的真理,而只求它能“管用”,回顾我们曾经说过的\f{如果一个传统关联“seems”和“is”的方式土崩瓦解,它就陷入了认识论的危机。},因此,工具主义是科学陷入危机的典型标志。

伽利略通过三重策略解决了这个危机:拒绝工具主义;调和天文学和力学;重定义实验在自然科学中的地位。

伽利略的优点在于,他指出了我们应当如何“诉诸事实”,即通过精心设计的,可控的,理性化的实验来检验理论,进而从一套共同的标准评估前人的工作。也就是说,伽利略重写了科学传统的叙事,中世界晚期的科学史得以被统一成一个连贯的叙事。我们可以识别哪些东西是理论的反例,哪些可以通过特设性的(Ad-hoc)解释来排除干扰,并观察各种理论中的不同要素在与其他理论和真实实验中的表现如何。

作者认为,“The criterion of a successful theory is that it enable us to understand its predecessors in a newly intelligible way(一个成功理论的标准是,它能让我们以一种新的可理解的方式理解其前辈\f{笔者注:比如,路易斯酸碱质子理论可以正确理解阿伦尼乌斯酸碱理论。或许这样我们才不至于在寻求真理的过程中丢失可理解性。}).”同时这一理论应当能让我们理解:

\begin{enumerate}
    \item 为什么其前辈必须被拒绝或修正\f{阿伦尼乌斯酸碱理论无法解释氨气溶于水显碱性。};
    \item 为什么在没有它之前,过去的理论基本保持可信\f{阿伦尼乌斯酸碱理论从电离出发,可以解决当时绝大部分常规的酸碱的问题。}。
\end{enumerate}

作者在这里提到了叙事与传统之间的关系:“For what constitutes a tradition is a conflict of interpretations of that tradition, a conflict which itself has a history susceptible of rival interpretations(传统是由冲突的解释所构成的,每一个冲突的解释都有其历史发展轨迹,而这个历史容易被对立的解释所影响). ”,也就是说,传统永远处于未完成的状态,是一个动态的,争议性的过程。

因此,一个传统不仅体现了一个争论的叙事,而且只能通过一种\b{论辩性的重述}(Argumentative Retelling)来恢复,而这种重述同样与其他重述相冲突。也因而每个传统总是陷入不连贯的危险,此时只有通过一次\b{革命性的重构}(Revolutionary Reconstitutioin)来恢复。保守理论家往往认为传统是和理性,革命对立的。然而,在作者的论证下,传统才是理性的承载者,并且传统需要通过革命以延续。

冲突不只发生在传统内部,也发生在传统之间。此时检验的是传统的生存力\f{此处作者用“Resource”一词,我们将其理解为一个传统内在的生存韧性。}。一个传统退化的标志是,它人为的设计了一套认识论的防御机制,使得它可以避免被质疑,或者至少避免认识到他正在被敌对的传统所质疑。

那么,回顾笛卡尔的行为。他隐瞒了他所用于表达思想的背景。这个背景让他可以表达出可理解的内容,假如这个背景也被质疑,人就会陷入“精神崩溃”。休谟就陷入了这样的崩溃中,他的怀疑论使得他不能理解和被理解。纯粹的经验主义很可能将我们置于一个毫无内容的现在,丧失所有普遍规律和对未来与过去的态度。

\subsubsection{麦金泰尔的科学哲学}

在前面的两章内容中,作者将认识论危机,叙事,传统,自然科学,怀疑论,疯狂等概念联系起来。这是作者对于当时库恩等人在科学哲学方面争论的回应。

现在,作者想要强调一个观点:戏剧性叙事\f{戏剧性叙事,就是以一个戏剧性的方式描述科学的发展,有其主角,冲突,转折,解决。}是理解人类行为的关键形式。当且仅当人们将书写真实的戏剧性叙事(即以特定方式理解的历史)视作一种理性活动时,自然科学才能够成为一种理性的探究形式。\f{Dramatic Narrative似乎在描述上比平淡的记叙更具科学性,因为戏剧性的叙事理清了事件发展的逻辑和评价。}这样,科学哲学最终从属于历史理性\f{历史理性:人类通过理解历史来把握世界意义、构建知识并指导实践的理性能力。它反对将理性视为抽象的、永恒的数学逻辑,而强调理性本质上是在时间中生成、受历史条件塑造的动态过程。},并且只在历史理性的语境下可以被理解。通过一个\textit{\b{fortiori}}\f{一个源自拉丁语的逻辑属于,意为“由更强理由”,核心逻辑是:如果某个结论在较弱条件下成立,那么在更强条件下必然更成立。},不难证明这一观点不止对自然科学成立,更对社会科学成立。

\q{为什么说对自然科学成立的观点可以由一个fortiori证明对社会科学成立呢?这种跨学科方法论是正确的吗?}
\ans{NULL}

接下来,作者将要提出他对库恩在科学哲学方面的批评。这一批评分为三个部分。
\begin{enumerate}
    \item 指出其早期立场的缺陷远比他所承认的严重得多;
    \item 正是由于他没能认清其早期表述中缺陷的本质,导致了他后期修订中的弱点;
    \item 提出一个充分的修订方式。
\end{enumerate}

库恩最初提出的科学哲学中的认识论危机的表述本质上与笛卡尔相同。这种表述要归功于迈克尔·波兰尼的著作。波兰尼认为所有\textit{\b{Justification}}\f{Justification指赋予信念,行为,主张以合理性的过程。将主观意见升华为有理性支撑的知识。}都发生在社会传统内部,这种传统常常强制执行未被认识到的规则,通过这些规则,不一致的证据碎片或困难的问题常常在科学界默契的同意下被搁置一旁。作者认为波兰尼是科学哲学领域的伯克,因为他将传统理解为保守的和单一的。所以任何继承了波兰尼观点的人都无法解释一个传统向另一个传统的过渡,因为理性似乎只发生在传统内部,不可能做到过渡和重构,过渡和重构必然是某种在黑暗中进行的跳跃\f{高考物理笑传之忽略边界条件。}。

波兰尼从未将他的论证推到上面这一点,但库恩却发现了这个困难。他不仅充分认识到科学传统如何可能陷入不连贯,并且指出了\b{不可通约性}(incommensurability)\f{由库恩提出,指不同理论,范式和文化体系之间缺乏共同的衡量标准,导致它们无法被直接比较或理性评判。这颠覆了传统“进步性累积”的观点,进步性累积(可通约科学观)认为科学是通过累积真理而进步,但不可通约性指出科学在范式革命中断裂式发展。}的意义和特征。

库恩的提出的不可通约性有三重前提,同时这三重前提也是\f{三重不可通约性},即对于科学革命期间,对立范式的拥护者:
\begin{enumerate}
    \item \b{问题标准不同}:他们对于检验成功范式的问题集存在分歧\f{比如,牛顿力学框架认为水星近日点的进动误差是可容忍的,但相对论却认为这是检验相对论正确性的核心证据。};
    \item \b{概念体系不同}:他们的理论体现了非常不同的概念;
    \item \b{知觉经验不同}:他们“从同一点向同一方向看时看到不同的东西”。
\end{enumerate}

库恩总结说,不可通约者之间的过渡不可能一步步完成,为此他引入了如\b{“格式塔转换(Gestalt switch)”}和\b{“皈依体验(conversion experience)”}这类概念。

库恩的知名观点是“proponents of competing paradigms must fail to make complete contact with each other's viewpoints(竞争范式的支持者如同生活在不同世界的人)”以及范式范式之间的过渡需要一种\b{“皈依体验(conversion experience)”}。然而,这种观点却并不是发轫于他的三重不可通约性,而是以波兰尼的立场为前提。

现在思考,从\b{三重不可通约性}抵达\b{非理性的皈依体验}有着一个隐性的前提,那就是:范式间的分歧彻底侵染了理性运作的每一个可能领域(不止问题,概念与知觉),这样,在范式转换中,理性将完全失效。这个额外的前提正是由于库恩继承了波兰尼的立场,预设了类似波兰尼的观点。

因此,库恩被他的批评者指为非理性主义者。

从这样的表述中,我们会发现科学革命是以笛卡尔方式理解的认识论危机,一切事物都同时受到质疑,理性在危机的边界失去了连续性,我们不可避免地会像帕斯卡那样,认为“The highest achievement of reason is to learn what reason cannot achieve(理性的最高成就是认识到理性无法成就什么)”。然而正如作者所说,笛卡尔的认识论危机观是错误的。永远不可能出现同时质疑一切的情况,因为那会导致所有事物变得彻底不可理解。

此外,库恩并没有区分两种类型的过渡:库恩的假设下,过渡的主体是生活在旧范式中的人,他必须进行某种已经由他人完成的过渡来进入新的范式。因此库恩总是描述着一个新的范式正在运作,旧的范式仍有力量的处境。但是还存在着最先发明和发现新范式的科学家,这些人将经历怎样的过渡呢?

一个被教育进入某种传统的人,当意识到其自身的认识论理想与实际实践有了差距后,就会陷入认识论的危机。此时,一些人会倾向怀疑论,一些人会倾向工具主义,而还有一部分天才则能在过渡中实现一种革命性的突破:发明新范式,同时重构对旧范式的理解。通过从新科学的立场出发,我们得以描述旧科学的不足,因此新科学比旧科学要更充分,叙事历史的连续性得以重新确立。

对于库恩,他一直在修改自己的早期表述。他不承认自己描述的科学革命是非理性的,并且认为“如果历史或任何其他经验学科引导我们相信科学的发展本质上依赖于我们先前认为是非理性的行为,那么我们不应得出科学是非理性的结论,而应得出我们的理性概念需要在此处或彼处进行调整的结论”。

费耶阿本德\f{费耶阿本德维护和论证相对主义,非理性主义与反科学主义,同时提倡认识论无政府主义,被认为是当代科学哲学中的最大异端。}从与库恩相同的前提出发,却得到了非理性主义的结果。如果科学革命真的如同库恩自己所认为的那样只具有三重不可通约性,那么我们必须承认费耶阿本德的正确性。而如果要像库恩所说,去调整理性的概念,则不得不在科学革命中找到某个尚未被关注到的特征,这个特征说明了理性在科学革命中的存在。

那么,存在这样的特征吗?这种特征存在于历史之中。考虑:只有当从新的范式的立场出发,我们能对旧范式的被接受,生命历程,到被拒斥能够以比以前更能被理解的历史叙事来讲述时,接受新的范式才是更理性的。也就是说,理解\b{一种物理理论优于另一种}的概念,需要预先理解\b{一种历史叙事优于另一种}的概念。科学理性将被嵌入一种历史哲学中。

从一个范式向另一个范式所传递的,是人们的认识论理想以及对于“什么构成了个人智慧进步”的理解。也就是说,我们“使自己被理解”的叙事,在历史上是连续的。

作者认为,自己正在接近拉卡托斯摆脱波普尔最初立场后的立场。波普尔曾提出证伪主义来说明科学的发展方式。其后继者亦常使用更充分的版本来取代波普尔最初的证伪主义。然而这些尝试无一例外都被其后的科学史证明是失败的。费耶阿本德认为,不存在关于科学必须如何进行的一套规则。拉卡托斯认为,评价更应针对一系列理论,而非一个孤立的理论。我们认为是单一理论的东西,始终是“一个成长发展着的实体,不能被视为静态结构”。在每一个阶段,一个理论都带有着其先前历史的印记,即一系列与确证性或反常证据,其他理论,形而上学观点等等遭遇的印记。评价它就是评价它在这一系列遭遇中的表现。评价一个理论,正如评价一系列理论,恰恰就是书写那段历史,书写那部关于失败与胜利的叙事。

作者主张,方法论(Methodologies)应根据它们在多大程度上满足史学标准来评估,而最好的科学方法论是能够提供对科学史的最佳\b{理性重建(rational reconstruction)}的方法论,并且对于不同的事件,不同的方法论很可能是成功的。但拉卡托斯没能摆脱波普尔主义传统中对“历史真实性”的轻视\f{也就是说,他们更在乎科学方法论的逻辑规范,而相对忽视历史是否符合这些规范。},他将理性重建设想为实际历史的艺术描述(Caricature)。但作者认为,我们的历史应当是真实的。正如科学理论应当将真理作为其目标之一一样。

库恩同样认为历史叙事必须真实,但他同时认为自然科学本身不能追求超越“解谜能力”和“具体预测”的真理,也就是说,科学理论不能提供真实的\b{本体论(Ontology)}\f{本体论:探究世界的本原或基质的哲学理论。}的内容,因为我们不能脱离理论框架去判断理论与实在是否相匹配。但作者对之嗤之以鼻,因为科学史明确证明,科学能够证实对某些存在的主张是假的(比如燃素,女巫和龙),同时还能确立分子/电子等实体的存在地位。

同时,库恩以爱因斯坦的相对论为例,认为其优越性仅在于比牛顿力学能解决更多/更好的难题,并非更接近关于真实的本体论。他甚至认为在某些方面,爱因斯坦理论比牛顿理论更接近亚里士多德理论。但作者指出,从接近本体论的角度看,亚里士多德力学不可能接近狭义相对论,而牛顿力学却可以,我们不可能从亚里士多德力学跳跃到爱因斯坦力学。

作者认为,库恩所真正忽略的,是不同科学领域通向真理的过程具有\b{趋同(Converge)}的特性。在科学史上,不同领域的科学理论在核心概念,解释框架和对世界构成的描述上,惊人地逐渐变得相互兼容,协调一致,并指向一个统一的,关于世界根本秩序的理解。如果科学不能够揭示本体论真理,这种趋同性不应当如此显著,不同的学科完全可以只关注与解决各自的谜题,发展出互不相容的实体。

库恩观点的吸引力仅仅在于它与\b{可错主义(Fallibilism)}的一致性。或许爱因斯坦物理学有朝一日会被推翻,或许正如拉卡托斯所说“我们所有的科学信念过去是,现在是,将来也总是虚假的”,但必须强调的是:可错主义必须与\b{趋近于对事物基本秩序的真实描述}这一\b{规范性理想(Regulative ideal)}相一致,而不是相反。我们期待宇宙有一种潜在的秩序。

作者给出的替代方案如下:能够解释为何某些科学理论更优的最佳方式,就是构建一个可理解的,有内在逻辑的\b{戏剧性叙事}。这个叙事具有历史真实性,并将这一系列理论统一成一个连续的故事。\b{正是并且仅仅是因为}我们能够构建这种叙事的更好或更差的版本,这些版本可以彼此进行理性比较,我们才能能够对理论也进行理性比较。

物理学理论预设了自己的历史,而科学的进步本质上依赖于对其自身历史的解读,这种解读涉及到作者提及的三个概念:
\begin{enumerate}
    \item \b{传统。}科学是积累性事业,新理论在旧传统的框架中诞生;
    \item \b{可理解性。}历史叙事必须揭示理论更替的合理逻辑;
    \item \b{认识论危机。}旧理论因内在矛盾或无法解决关键问题而崩溃,进而为新理论提供契机。
\end{enumerate}

通过这样的方案,我们调和了库恩和拉卡托斯的观点争执。库恩被拉卡托斯所批评的非理性主义,实际上是忽视了科学革命中历史叙事的连续性与理性逻辑,缺少了拉卡托斯对于历史叙事的那份重视;拉卡托斯被库恩批评为“逃避历史”,实际上是没有把“理性重建”嵌入真实的历史叙事。

\subsubsection{总结}

现在可以阐明我们的最终论点了:如果我们忽视叙事与传统的联系(科学理论的发展是嵌入在历史叙事和学科传统中的连续过程),理论与方法的联系(评价理论的方法论规则不能脱离理论所处的具体历史语境),科学哲学就将面对无法解决的问题。任何有限的观察数据(无论多庞大)在逻辑上都与无限多个可能解释这些数据的普遍理论或假说相容\f{我们提到过休谟问题,也即归纳问题,其指出归纳法缺少理性基础,因为有限观察在逻辑上无法必然推出普遍结论,比如白天鹅的归纳问题,同时归纳法的合理性只能通过“过去的归纳有效”来归纳证明,是一种循环论证。}。任何试图通过提供一套规则来一劳永逸地证明科学的理性,以连接观察数据和可普遍化的假说的尝试都失败了。

事实上,只有将科学理论置于其产生的具体历史语境中考察,才能判断接受或拒绝之是否合理。否则我们将陷入教条主义\f{强行套用僵化的方法论规则来判决历史,扭曲事实以适应规则(如拉卡托斯的理性重建)。}或是怀疑论\f{因找不到普遍规则而否定科学有任何合理性。}的困境。

因此,结果表明,从十八世纪开始,科学哲学那种结合经验主义和自然科学的纲领,要么最坏地崩溃于非理性主义,要么最好地崩溃于一系列相继弱化的经验主义纲领\f{面对根本性困境时,经验主义不断弱化自己的主张,通过分析哲学的技术性修补逃避现实。但正如作者所说,休谟的疾病是无药可救的,无论如何弱化也无法逃避归纳问题。}(这种经验主义的驱动力是一种深刻的,不愿被迫走向非理性主义结论的愿望)。

最终,我们会发现,在理解历史与物理学的关系时,维柯\f{历史主义先驱,主张“真理即创造”,即人类最能理解的是自己创造的东西——历史和社会世界(而非物理世界)。理解科学必须理解其作为人类创造物的历史发展过程。},而非笛卡尔(代表理性主义,忽视经验和历史)或休谟(代表经验主义,强调经验是知识唯一来源,但揭示出其无法为科学提供逻辑基础)才是正确的。理解物理学本身,必须将其视为一种处于历史进程中的,由特定文化传统和实践塑造的人类活动。

\subsection{《安斯康姆论实践知识与善》}
\textbf{Anscombe on Practical Knowledge and the Good, Frey}
\subsubsection{引言}
伊丽莎白·安斯康姆师从维特根斯坦,是有史以来最重要的女哲学家之一,在伦理学方面有重要贡献。《意向(Intention)》和《现代道德哲学(Modern Moral Philosophy)》是她的两部极具影响力的著作,前者是\b{行动理论(action theory)}\f{行动理论主要探讨人类有意识行为的本质、原因、过程和结果,它试图回答一个核心问题:人为什么会做出某种行动?这些行动是如何发生、如何被理解、又如何影响世界的?}的里程碑式文本,后者在伦理学上有着同样的影响力。作者在本文意图探讨这两部作品的相互关联。

在《现代道德哲学》中,安斯康姆建议我们在拥有“充分的心理学哲学”以解释“行动”“意图”这类术语之前,“完全把伦理学从我们心中驱逐出去”。可见,安斯康姆认为行动理论在某种意义上是伦理学的正当基础。

在《意向》中,有两个重要论点:
\begin{enumerate}
    \item \b{行动解释的知识要求(Knowledge Requirement)。}意图性行为必须以一种“直接且非证据式”的方式为行为者所知;
    \item \b{行动解释的善性要求(Goodness Requirement)。}行动总是在“善的表象”(guise of the good)下被追求,即行为者将其所为视为追求某种善。
\end{enumerate}

许多人猛烈抨击这两个观点,或者只接受其中之一。但弗雷强调实际上这两个要求是共生的。因为在安斯康姆的解释下,\b{知道自己在做什么}与\b{知道做此事所意图的善}是\b{实践性自我认知(Practical Self- 
knowledge)}\f{实践性自我认知强调对自身动机、意图、信念、情感和行为模式的深刻理解,并以此指导实际行动、决策和生活方式。其聚焦于如何运用这种自我认知来更理性、更有效地生活,实现个人目标,并提升整体福祉。}的一体两面。

作者将在安斯康姆“呼吁停止道德理论知道我们理解更基础的行动理论”这一背景下思考《意向》中的释经学与哲学问题。

\subsubsection{规范问题与实践知识的标准解释}
在这一部分,作者讨论《意向》的核心问题:\b{规范问题(Specification Problem)}——即区分行动的有意(Intentional)描述和无意(Unintentional)描述。

考虑:对于任何行动,总是存在许多真实描述,而这些描述不都是有意的。我们想要知道,哪些描述是有意描述?行动理论必须提供一个答案,因为法律和道德都需要依赖于一个原则性的方法,去区分人通过能动性有意去做的事和仅仅是因为运用能动性的结果所导致的无意的事。这就是\b{规范问题}\f{规范问题是关于行动意向性的问题,想要判断行动的好坏,我们必须能够准确的描述他是哪种行动。假设我走进一个房间,看到某人“将刀刺入史密斯的大腿”。这个描述层次对于道德评价来说还不够具体。我是目睹了一个被恰当地描述为“自卫免受非法攻击”的人,还是目睹了一个“犯谋杀罪”的人?}。

安斯康姆认为,一个行动是否是意向性的,关键在于我们能否有意义的提出一个“为什么”的问题。当对于某个描述提出这个问题时,如果行动者拒绝解释这个行为的理由,就可说明这个描述是一种无意描述。

Johns想要开灯,但是开灯的行为吵醒了狗。我们描述:Johns吵醒了狗。我们问他,“为什么要吵醒狗”,他一定会拒绝解释这个问题。可见,这个描述是一种无意描述。

“为什么”是一个针对理由的问询,我们不能在无意的情况下回答这个问题,也就是说,当“无意”时,我们无法给出理由。这意味着,\b{我们意识到自己的一个行动概念}与\b{有做这一行动的理由}是有关联的。安斯康姆将其解释为知识角度的关系:如果Johns不\b{知道}他打开开关这一行为吵醒了狗,那么他显然是无意的。也就是说,对行为的无意描述往往从第三人称、观察者的视角进行,并且描述了一个行动者完全没意识到的行为。如果一个人仅能从第三人称的途径\b{发现}他正在做某事,那么他一定(eo ipso)不是有意为之。一个有意描述必须从第一人称视角进行,并且总是\b{“模糊且不确定的(vague and indeterminate)”},

\section{Tuesday}
\subsection{《物理理论与自然分类》}
\textbf{Physical theory and Natural Classification, Duhem}\f{选自《物理理论的目标与结构(The Aim and Structure of Physical Theories)》Part1-Chapter 2。}

\subsubsection{物理理论的真实本质和构成操作是什么?}

当我们把物理规则当做对\b{物质实在(Material Reality)}的\b{假设性解释(Hypothetical Explanation)}时,我们便使其依赖于\b{形而上学(Metaphysics)}\f{形而上学由亚里士多德提出,是指对世界本质的研究,即研究一切存在者,一切现象(尤其指抽象概念的原因及本源)。}。但这会导致不同意形而上学的人不会认同物理理论,况且物理理论似乎并非完全从形而上学学说中获得原则性的内容。

因此,我们自然地想要问两个问题:
\begin{enumerate}
    \item 我们能否赋予物理理论一个自主性的目标?使物理理论完全同形而上学学说分离,并用其自己的术语来评判自身,以免受物理学家哲学学派不同带来的影响。
    \item 我们能否构想一种足以建构物理理论的方法?这种方法与物理理论的自身定义保持一致,其不使用任何非法的原则,也不诉诸(Have no recource to)任何非法的程序。
\end{enumerate}

Duhem立即给出一个定义:\b{物理理论不是一种解释性内容。它是一个数学命题系统,由少数几个法则(Principles)推理而出(Deduce),旨在尽可能简要、全面、精确地表明一组实验规律。}

为了精确化这个定义,他指出了构成物理理论的四个连续操作:
\begin{enumerate}
    \item \b{定义和测量物理量值。}我们首先选择简单的物理属性(Physical Property),这些属性可以组成其他未被直接定义的属性。接着,通过适当的测量方法,我们关联起数学符号和物理属性。此时符号和属性之间没有本质联系,我们只是通过这种方式,使得物理属性的与对应代表符号的一个值联系起来。
    \item \b{选择作为演绎基础的假说。}我们借助少量的命题(Propositions)来关联引入的不同种类的量值(Magnitudes),然后将这些命题作为我们演绎推理的基本法则。这些法则可以被称为“假说(Hypotheses)”。假说是理论构建的基础,但不会被要求表明物体真实属性之间的真实关系,因此其可以任意表述,只是不能存在逻辑矛盾。
    \item \b{进行数学演绎。}我们通过数学分析的规则组合理论的各种法则与假说,此时我们只需要要求代数逻辑的合理性。所有操作是否对应真实的物理变化是不重要的,我们只要求:1.用于假说演绎的三段论推理有效;2.计算无误。
    \item \b{比较理论与实验。}通过验证假说,我们得到各种推论,接下来,如果推论符合实验现象,我们就认为这是一个好的理论。也就是说,实验相符是真理性的唯一标准。
\end{enumerate}

\q{这是否带有一种工具主义倾向?如果是,Duhem是否面临一种认识论危机?}
\ans{我想我们当然可以认为Duhem带有一种工具主义倾向,但回顾认识论危机是什么:我们遇到两种相对立的叙事,使我们混乱进而拒绝相信有一个对客观真实的表述,这会导致一个传统陷入不连贯的危机。然而,在Duhem的观点下,一个物理理论(tradition)具有解释性的部分和表征性的部分,这个表征性的部分在传统延续中并不会陷入危机,而是被传递下去;真正面临冲突的是解释性部分,而这个解释性的部分会被更优的叙事代替。Duhem实际上拒绝了工具主义对本体论的绝对否定,他一定程度上相信表征性理论可以成为一种对本体论秩序的逼近。}

\subsubsection{物理理论有何效用?理论乃思维经济}
我们通过数学命题系统形成物理理论,而这种理论用数量极少的命题代替了大量看似独立的定律。一旦假说已知,我们就可以通过数学演绎获得所有物理定律,这极大地降低了我们的思维负担。

物理定律向物理理论的还原促成了Ernst Mach所认为的科学目标和指导原则——\b{思维经济(Intellectual Economy)}。通过实验得到的实验定律(Experimental Law)本身就是一次思维经济的获取,我们将数量庞大的具体实施抽象成普遍的东西,即物理定律(Physical Law)。当我们把其浓缩为理论时,这种思维经济得到倍增\f{想象在光学领域,我们可以用折射定律(物理定律)描述所有的光的折射现象(实验定律)。而对于折射、反射、双折射、干涉、衍射、旋光偏振、反射偏振等等物理定律,我们又构造出一个光学理论(物理理论),将这些定律浓缩为少数几个法则,并且从这些法则出发,我们可以通过数学演绎提炼出我们想要使用的定律。}。这正是物理科学前进的方式:实验事实 → 定律 → 浓缩理论。

\subsubsection{理论乃分类学}
理论不仅仅是实验定律的经济性表征(Economical Representation),同时也是对这些定律的一种分类。

面对实验物理学,我们会得到一堆同一层面的定律,而没有对其的分组。而理论通过发展链接法则与实验定律的演绎推理过程,建立了一种秩序和分类。通过分类,我们能确定无疑、不加增删的找到解决特定问题的定律。

正如作者所感叹的那样:“It is impossible to follow the march of one of the great theories of physics, to see it unroll majestically its regular deductions starting from initial hypotheses, to see its consequences represent a multitude of experimental laws down to the smallest detail, without being charmed by the beauty of such a construction, without feeling keenly that such a creation of the human mind is truly a work of art(但凡追踪一门伟大物理理论的发展轨迹,目睹它从初始假说出发,庄严地铺展其规整的演绎推理,见证其推论竟能细致入微地摹写无数实验定律 —— 便无法不被这一构造之美所俘获,无法不深切体悟:人类心智的这一创造,实为一件真正的艺术杰作).”

\subsubsection{理论趋于变为自然分类}
审美情感(Esthetic Emotion)并非一个高度完美的理论所产生的唯一反应,一个理论还向我们展示出一种自然分类。

首先,我们定义自然分类(Natural Classification)。这里笔者倾向于用自己的语言做概述,即自然的分类是这样一种分类:虽然我们的知识无法证明这种分类是符合真实的,但构建分类的过程是如此的巧合且巧妙,形成了极明晰的排序,压倒性地说服我们这样的分类并非纯粹的人为,我们无法解释分类背后的实在,却强烈的感受到,这种分类反应出事物之间本质而真实的关系。这样的分类就是一种自然分类。

我们引用作者的总结:“Thus, physical theory never gives us the explanation of experimental laws; it never reveals realities hiding under the sensible appearances; but the more complete it becomes, the more we apprehend that the logical order in which theory orders experimental laws is the reflection of an ontological order, the more we suspect that the relations it establishes among the data of observation correspond to real relations among things, and the more we feel that theory tends to be a natural classification(因此,虽然物理理论从未给予我们对于实验定律的解释,也从未反应出可感现象之下的实在,但当其愈发完备,我们就愈发地领悟到理论在规整实验定律的逻辑顺序时,所反映出的那种本体论的秩序,愈发猜想理论在观测数据中建立的联系,恰对应着事物之间的真实关系,并愈发的感到:理论正趋向于成为一种自然分类). ”

作者引用了Pascal的名言做结:“We have an impotence to prove, which cannot be conquered by any dogmatism; we have an idea of truth which cannot be conquered by any Pyrrhonian skepticism(“我们有一种无法证明的无能,任何教条主义\f{教条主义:固执地坚持某种信念或原则为确定的真理。}都无法将其克服;我们有一种真理的观念,任何皮浪式怀疑论\f{皮浪式怀疑论:“悬置判断以追求心灵平静”,不否定现象,但否定对现象的真理性解释,即不做任何确定的断言。}都无法将其征服。”).”

\subsubsection{理论预测实验}
理论包含一系列定律,这些定律可以被理论的推论所表征。但同样的,理论可以推导出无限数量的推论,我们大可以推导出一些从未见过的实验定律,而如果这种定律是可以实现的,我们就能通过事实检验这个理论\f{比如:泊松亮斑。}。

相应的,如果我们相信一个理论是自然分类,我们一定期待并理解这一理论能预测新的经验性内容、并刺激新定律的发现,因为这个理论应当表达了某种事物之间深刻而真实的关系。因此,\b{要求理论预测未来,就是对理论是否为自然分类的最高检验}。

\subsubsection{总结}
正如我们定义的那样:物理理论为大量实验定律提供了一个浓缩的、思维经济性的表征。它对这些表征进行分类,使其更容易、更安全的被使用。同时,物理理论带来一种整体的秩序,增加了美感并呈现出自然分类的特征,使其暗示了一种事物之间真实的亲缘关系,这种特征尤其通过理论的\b{丰产性(Fruitful)}来彰显,因为理论能预测尚未观察到的实验定律并促进其发现。

作者认为:“That sufficiently justifies the search for physical theories, which cannot be called a vain and idle task even though it does not pursue the explanation of phenomena(这充分证明了对物理理论的探索是合理的;即便这种探索并不追求对现象的解释,它也不能被称作徒劳无意义的工作).”

\subsection{《表征性理论与物理学史》}
\textbf{Representative Theories and the History of Physics, Duhem}\f{选自《物理理论的目标与结构(The Aim and Structure of Physical Theories)》Part1-Chapter 3。}
\subsubsection{自然分类与解释在物理理论演化中的作用}
在Chapter2中我们提出,物理理论的目标是成为自然分类,但这立刻引起一个反对意见:与其希望通过构建物理理论创造事物本体论秩序的一个投影,为什么不直接研究理论所暗示的实在本身呢?

必须承认,寻求对自然现象的本源解释可能激发了物理理论的发现。但我们必须证明,理论家所构想的这种解释性的理论的确构成了通向本体论的导向性内容,而不只是他们自己一厢情愿的臆想。换言之,我们需要证明:解释性理论是自然的而非人为的,这显然不可能。

我们分析一个物理理论,会发现它被分为两个部分:一部分是\b{纯粹表征的(Simply Representative)},旨在进行定律的自然分类;另一部分是\b{解释性的(Explanatory)},旨在把握现象之下的实在。人们往往认为解释性部分是表征性部分存在的原因,但这种关系是脆弱的、人为的。实际上表征性部分通过物理学特有的、自主的方法独立发展,而解释性部分寄生其上。理论中任何正确的东西都存在于表征部分,而错误和矛盾首先存在于解释性部分,归因于物理学家受到对把握实在的欲望的蛊惑而引入的错误。凭借着一种传统的连续性,解释性的部分不断被替换,而表征性部分近乎完整的进入新理论\f{笛卡尔发现的折射定律是一种表征性的理论,而他对光效应的解释性理论是这样的:光并不存在,而是一种精细物质(Subtle Matter)的快速运动所产生的压力,这种物质不可压缩,因此光瞬时传播,笛卡尔宣称,如果光不是瞬时传播,他宁愿承认自己的哲学被完全推翻。不难想象笛卡尔对光现象的解释性内容很快就完全崩溃了,光的波粒二象性学说曾在历史上此消彼长,但无论如何,折射、衍射、康普顿效应这些理论保留了了下来,而那些关于光现象的解释总是本身脆弱而逐渐衰老。}。

对于大多数物理学说,持久(Lasting)且丰产(fruitful)的总是那些逻辑工作;易逝(perishable)且贫瘠(sterile)的总是那些解释性的工作。科学像\b{涨潮(mounting tide)}般演化,海浪表面上往复运动,是对那些解释性内容的写照,而往复运动之下,海水持续稳定地向同一方向渐进运动,使得大海不断上升,这正是那些保证了物理学说传统连续性的表征性内容。

\subsubsection{物理学家关于物理理论之本质的观点}
这一部分文章主要讲述了物理学史的内容,并没有提出更多更重要的观点,不做赘述。但我们应当注意到的是,Duhem对物理学史的重视让我们想起MacIntyre的著名论断:理解\b{一种物理理论优于另一种}的概念,需要预先理解\b{一种历史叙事优于另一种}的概念。科学理性将被嵌入一种历史哲学中。

\section{Wednesday}
\subsection{《作为科学理论的实践思想》}
\textbf{Scientific Theories as Practical Thought, Nicholas Teh}\f{副标题:An Aristotelian-Anscombean reading of Duhem's \textit{The Aim and Structure of Physical Theories}(对多恩《物理理论的目标与结构》的亚里士多德-安斯康姆式解读),下文简称书名为AS。}

\subsubsection{引言}
本文的目标是勾勒一种思考科学活动(Scientific Activity)的方式,这种方式可以融入更广泛的亚里士多德式的实践思想(Practical Thought)与行动的图景中。

\subsubsection{亚里士多德式的实践思想}
首先,作者提到了两种不同形式的实践思想:
\begin{enumerate}
    \item 行动的目的(End)其赋予子行动(Subactions)以统一性(Unity)从而提供其形式(Form);
    \item 行动会显示出(manifest)某种人性之善(Human Good)或者某种实践之善(Goodness of Practice)。
\end{enumerate}

对于Anscombe的实践思想理论,\b{意向性行动(Intentional Action)}恰恰就是我们拥有(自我意识的)实践知识的那种行动,并且这些行动是实践上合理的(Practically Reasonable),以此理论指导的实践推理(Practical Reasoning)具有这样的形式:
\[
C-ing \xrightarrow{in\ order\ to} D-ing \xrightarrow{in\ order\ to} E-ing \xrightarrow{in\ order\ to} F-ing
\]
这正是我们所说的第一种实践思想形式。每一个子行动都被纳入一个更具包容性(inclusive)的行动中,知道我们抵达最终也是最具包容性的行动F,F给予我们这一系列行动的目的。Frey告诉我们:目的赋予其子行动以统一性,因为子行动都将是目的的构成性部分(Constitutive Part),对于这些构成性部分,如果不参照整体,就不能被准确的识别。一旦目的达成,这些子行动的统一性就将消失。

以上概念是对于意向性行动的一种\b{形质论(Hylomorphism)}\f{形质论:强调物质基础与本质结构统一性的学说,是形而上学体系的基础。}:子行动是行动的“质料”,而行动的“形式”由目的给出。在阿米巴原虫中,有丝分裂是繁殖;但对于人类它却是生长。类似的,同一个孤立的子行动描述,根据其所被纳入的更包容的行动及目的,将具有截然不同的特征。

对于Anscombe,她还认为意向性行动的目的,对于行动者来说,在一定意义上是“善的或可欲的(Good or Desirable)”。Frey则进行了更进一步的阐释:
\begin{quote}\small
The rational intelligibility of the practical determination of means or ends depends upon its inferential connection to the wider context of a life in progress, which has been and will continue to be shaped by one's general sense of how to live.
\end{quote}

这引导我们思考第二种实践思想形式,一种已经被Thompson和Rödl理论化的形式。回想,当目的完成时,行动的统一性就消失了。比如,我想修理我的自行车,一旦我修好自行车,这种欲望就停止了,它在解释它所解释的事情时耗尽(Exhaust)了自身。Rödl将这种\b{有限目的(Finite End)}与一种更广泛的\b{非有限目的(Non-finite End}区别开来。后者与某种实践的善相关,并最终与构成人类繁荣的那些善相关。可以说,这些善塑造了行动者“如何生活的总体意识”。

Rödl以健康为例,我们会为健康做各种事情,但似乎永远不会“完成”我的健康,只要我么坚持这个目的,它就不会耗尽自身。换句话说,此时我的各种行动并不是健康这个目的的构成性部分,而是为了\b{显现(manifest)}健康这个目的,这种显现并不会穷尽“健康”这一人类之善(Human Good),而这种善又反过来为有限目的的可欲性提供基础,这些有限目的会容纳一系列子行动。

于是,我们回顾了Anscombe所发展的那种亚里士多德传统中的两种实践思想形式。分别对应于“有限目的(赋予子行动统一性,在目的达成后消失)”和“非有限目的(善通过行动者无限多的行动显现出来,取之不尽)”。

\subsubsection{Duhem在AS中对科学实践思想的描述}
AS分为两部分,第一部分讨论物理理论的目标(Aim),第二部分则涉及其结构(Structure)。这种形式-目的的二分法(Bipartite)让我们想起上一节中对实践推理的思考。Duhem对科学进行了明显的目的论的说明,尽管该著作最吸引哲学界的整体论(Holist)论点——没有一个理论假说可以被孤立的检验——并没有以一种与这份目的论构想相关的方式被接受。

本节中,作者希望通过借鉴AS的第二部分,解释Duhem对物理理论结构的解释是如何契合Anscombe的实践推理图示的,并进一步启发我们:所谓Duhem的整体论,实际仅仅是关于实践推理不确定性的一般性洞见的一个推论(Corollary)。

仍然考虑那种:$X-ing \xrightarrow{in\ order\ to}Y-ing$ 的逻辑形式,这里Y是指构建某种科学表征(Scientific Representation),X被视作构成Y的一个子行动。

在Duhem看来,X代表对事物的具体处理(manipulations),Y是对相关物理量进行符号和数学描述的行动。这种实践思想的关键在于将一个具体描述X置于理论描述Y之下——这就迫使我们考虑将理论描述与具体事实匹配时会遇到的不确定性(比如测量方法、仪器精度带来的误差),这种不确定性将由科学家的实践判断来解决。

同时Duhem也考虑了这种推理的反向过程:在Y的层面上进行一些数学推演,以便将其应用于某个具体的实验设置X,他强调要审慎地(Prudential)判断Y是否是X的表征,一个偏微分方程(PDE)可能由于不适定性(ill-posedness)\f{适定性(well-posedness)是指方程的解满足物理过程或数学问题所需的基本合理性条件。一个PDE当且仅当其满足存在性、唯一性、稳定性(解连续依赖于初始条件或边界条件)才是“适定的”。}而无法表征物体的运动,因此不适合用作经验表征。

接着,Duhem着手处理一个更为熟悉的行动单元:将Y视作“进行一项实验(performing an experiment)”。在这个过程中,我们通过实践理智(Practical Intellect)将诸多数据和具体事实X纳入对实验的理论描述Y下,也就是说,我们可以用单一的命题来表达不同的事实,这些事实实际上是同一实验的不同形式。一个Y往往具有许多不同的构成性手段X,我们通过实践理智将X恰当的纳入Y。

Duhem进一步认为这种实践推理存在于实验的各个环节:比如,如果不将具体仪器X纳入一种“理想仪器”的理论的描述Y下,人们甚至无法获得一个关于“仪器”的科学概念。

Teh认为以上内容全部是为了第六章的主旨铺垫。第六章设计了“整体论”的陈述:特定假说无法被孤立检验。如果一个实验失败,只能说明我们用于预测现象的命题群中至少有一个错误,却不能知道这个错误出现在哪个命题里。而在Teh看来,如果用一种Aristotelian-Anscombean的方式解读,这种整体论观点实际上只是Duhem对科学中实践思想总体方法的一个推论,是对将X纳入Y时所产生的不确定性内容的延伸,由于我们应用了具有不确定性的实践推理过程,最终导致了整体实验的不确定性。

作者推广了Frey关于有丝分裂在不同生命形式背景下具有更广泛意义的这个例子,将\b{“生命形式(Life-form)}用于描述一种总体实践思想下的统一性,也就是科学理论特定的研究传统、概念框架、操作规范等内容。对于Duhem所谓的”特定假说不能被孤立证伪,不应被归结于“否定后件式和前提的合取\f{否定后件式(modus tollens)指的是通过逆否命题验证假说,前提就是说我们的一系列辅助理论:比如显微镜是好的,样本处理符合标准等,传统整体论认为我们不能孤立证伪的是二者的合取。笔者不禁感叹汉语竟然还能这样排列组合。}”,而是在于:我们所讨论的某个特定假说总是某种生命形式的一部分,这种生命形式就是被讨论的特定科学理论的总体实践思想。

我们想起,关于实践思想的第一个观点是:行动的目的赋予子行动以统一性,以及子行动的特征决定于更广泛行动的目的。所以,Duhem所说的那种整体论的内容,可以被理解为是理论所赋予子行动的统一性,当一个特定的假说被证伪,实际上被证伪的是他的一系列子行动;当一个定律被纳入另一个更大的定律,这个被纳入的定律就会获得一种新的身份,这个身份由更大定律更广泛的表征目的赋予。

\q{无敌了,完全理解不了}

对于实践思想的第二个观点,我们要考虑Duhem关于“自然分类”的观点,这也是Duhem关于科学理论终极目标的论述。Duhem认为物理理论通过连续的进步,将实验定律排列分类,逐渐逼近本体论秩序,于是我们不妨将这种自然分类视作一种非有限目的——任何功能良好的科学理论都将显现对自然的理解,但这种理解的目的无法被创造科学理论的互动所穷尽。

\subsubsection{结论}
作者认为,物理理论的统一性在于一种实践推理的统一性,并类似于一种实践或一种生命形式的统一性。最后提出,Duhem非常重视讲述叙事史的能力,让我们想起MacIntyre的著名论断。想要更深入的研究Duhem的思想,需要更深入的理解实践科学思想与Duhem试图提供的叙事解释的关系。

\section{Thursday}
\subsection{《因果性与决定论》}
\textbf{Causality and determination, Anscombe}
\subsubsection{因果性}
开篇第一句话我就非常绷不住,以至于我忍不住要引用他神秘至极的原文来让读者感受到这句话有多难以理解:
\begin{quote}\small
It is often declared or evidently assumed that causality is some kind of necessary connection, or alternatively, that being caused is-non-triviallyinstancing some exceptionless generalization saying that such an event always follows such antecedents. Or the two conceptions are combined. 

人们经常宣称或显然默认:\b{因果性(causality)}是某种必然联系。或者换一种说法,“被引起”——非平凡的——是某种无例外普遍概括的实例化,该概括声称此类事件总是跟随此类先行条件发生。或者这两种观念被结合在一起。
\end{quote}

不可否认,笔者的英语水平低劣到令人发指的地步,但是面对这些句子的汉语翻译,我仍然要说,中文竟然还能被这样排列组合,我还是孤陋寡闻了!

现在我们理解这个句子:考虑有事件A和由事件引起的事件B,显然A就是一种\b{“先行条件(Antecedents)”}。\b{无例外普遍概括},就是说该条件不允许任何例外(无例外),覆盖所有符合条件的情形(普遍),描述了事件类型之间的关系而非具体事件(概括),而我们要对这种概念进行“实例化”,才能使其对应具体的情形。所以,事件B“被引起”总是跟随A这个先行条件而发生,而B被A引起的因果过程一定是某种满足“无例外普遍概括”的铁律的实例化。现在我们只剩下“非平凡的”这个词要处理,在哲学中,“平凡的”是指:
\begin{enumerate}
    \item 仅通过定义或语言形式就能成立,无需经验验证;
    \item 不提供任何新信息,只是重复已知的逻辑或语言规则;
    \item 偶然成立但无普遍意义的关联。
\end{enumerate}

所以非平凡的是指,我们对因果性的定义是事件之间的真实关联,而不是空洞的内容、偶然的顺序。

综上,我们就可以理解“因果性是某种必然联系”的这个“换一种说法了”。

我们回到原文:如果我们认可这种必然联系,则如果相似的情形没有引发类似的结果,则一定存在一个差异。然而,对于对立与这种绝对因果性的观点,往往会拒绝这一假说。尽管我们承认,在有些时候,假设存在这样的差异是研究的可靠原则;但是这种差异绝不应该被认为是在及其普遍的意义上存在的。

事实上,这种将因果性视作必然联系的观点是过去许多哲学家的观点,通常以\b{“新休谟主义(neo-Humean)}的形式被接受,这种世界观在我们的思维方式中根深蒂固。作者提到Aristotle、Spinoza(斯宾诺莎)都有类似的描述。但尤为重要的是Hobbes(霍布斯)的一个表述:
\quo
{A cause simply, or an entire cause, is the aggregate of all the accidents both of the agents how many soever they be, and of the patients, put together; which when they are supposed to be present, IT CANNOT BE UNDERSTOOD BUT THAT THE EFFECT IS PRODUCED at the same instant; and if any of them be wanting, IT CANNOT BE UNDERSTOOD BUT THAT THE EFFECT IS NOT PRODUCED.

一个简单的原因,或一个完整的原因,是所有施动者(无论有多少)和所有受动者的偶性(accidents)的总和;当它们被假定在场时,\b{无法理解结果不在同一瞬间产生};如果它们中任何一个缺失,\b{无法理解结果不被产生}。
}

这种观点将因果视作一种逻辑必然,并在之后被Hume推翻。Hume让我们看到,给定原因及其结果,通常不存在逻辑上的矛盾,让我们不能假设一个发生而另一个不发生。事实上,人们早就习惯于相信奇迹、奇观、自然的戏谑(lusus naturae)\f{这个名词常用来指自然界那些反常,难以用常规规律解释的现象,比如畸形的生物、罕见的自然奇观。},只是过去的哲学家过度逻辑化因果性,导致人们并没有意识到这些观点早就推翻了因果所被赋予的那种逻辑联系。

然而,Hume不仅没有否认因果性与必然化的等同,反而还加强了它。事实上,文章开头那个晦涩的句子就是Hume的核心观点,因果性之必然联系的本质实际上在于人类心灵中那些\b{恒常联结(Constant Conjunction)}的经验性内容,由此暗示了因果性与决定性定律(Deterministic Laws)\f{给定初始条件和定律,总是决定一个唯一的结果。}之间的联系。

不难看出,Hume似乎把因果性看做一种经验主义的内容。Kant(康德)于是开始寻找反驳Hume的方法,努力证明因果性实际上是一种先验(priori)的概念。在他的论证中,客观的时间秩序实际上是“in that order of the manifold of appearance”,显像(Appearance)是未经知性整理的感性材料之和,它们处于一种杂多(Manifold)的状态,由知性赋予其秩序,这种秩序告诉我们,我们先领会之前的事,才能领会之后发生的事,而先前的事中\b{必然且不变的(INVARIABLY and NECESSARILY)}存在一个事件的条件。这也是反对休谟的人的普遍特征:先验地,或者以某种方式从经验中确立因果性中那种必然化。

\q{我们在这里似乎遇到了一个问题:休谟的观点不是因果性是一种经验主义吗?从经验中的到必然化怎么能反对休谟呢?}
\ans{事实上,休谟认为经验中只存在着一种“恒常联结”,经验告诉我们,A往往跟随B发生,但不能告诉我们,B必然导致A,所谓的必然性只是因为我们总是在B后看到A所导致的主观产物。我们提到,休谟的这种观点暗示了决定性定律,这是因为科学定律往往是一种“无例外普遍概括”,或者说是一种“无例外”的“恒常联结”,由于无例外,我们的心灵才把它视作绝对确定的信念,这也可以视作对因果性的一个“普遍性”的解释。而批判他的经验主义者则会认为,经验中所提取到的“无例外的普遍规律”并不是心灵的虚构,而恰恰就是对因果性必然化的证明。总的来说,二者的分歧在于,休谟认为必然性只是在经验下心灵对一种“恒常联结”的主观推断,但批评者会认为经验呈现了一种客观的“必然化”。}

我们常常用充分、必要条件来解释因果性,充分条件是一个被定义好的术语,我们可以不去质疑“
Russell(罗素)曾谈及因果律,认为人们错误地视这种观点为无害。他对必然性(强条件,A必然导致B,逻辑或客观必然)的概念表示怀疑,除非用一种普遍性(Universality,弱条件,在情境中B总是跟随A发生,范围上无例外)来解释之。

\subsubsection{决定论}

想象一个球的运动,我们可以用慢镜头播放它以展示每一阶段的运动。在牛顿力学未被质疑的时代,我们会强烈的相信物体接受一种运动上的“绝对命运(Absolute Fate)。然而由于测量的不确定性,加之微小的不精确因子的互作,我们实际上难以得到一个精确的结果,而只能确定某一个结果的概率。对于一个极高的概率,我们相信它有一种实践上的确定性(Practical Certainty)

\end{document}